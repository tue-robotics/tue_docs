Tech United Eindhoven\footnote{\url{http://www.techunited.nl}} (established 2005) is the RoboCup student team of Eindhoven University of Technology\footnote{\url{http://www.tue.nl}} (TU/e), which joined the ambitious @Home League in 2011. The RoboCup@Home competition aims to develop service robots that can perform everyday tasks in dynamic and cluttered `home' environments.
The team has been awarded multiple world vice-champion titles in the Open Platform League (OPL) of the RoboCup@Home competition during previous years, and two world champion titles in 2019 and 2022\footnote{\url{https://tinyurl.com/DSPLBangkok2022Stage1Score}}\footnote{\url{https://tinyurl.com/DSPLBangkok2022Stage2Score}}\footnote{\url{https://tinyurl.com/DSPLBangkok2022FinalScore}} in the Domestic Standard Platform League (DSPL). In the DSPL, all teams compete with the same hardware; all teams compete with a Human Support Robot (HSR), and use the same external devices. Therefore, all differences between the teams regard only the software used and implemented by the teams. \\

\noindent Tech United Eindhoven consists of (former) PhD and MSc. students and staff members from different departments within the TU/e. This year, these team members successfully migrated the software from our TU/e built robots, AMIGO and SERGIO, to HERO, our Toyota HSR. This software base is developed to be robot independent, which means that the years of development on AMIGO and SERGIO are currently being used by HERO. Thus, a large part of the developments discussed in this paper have been optimized for years, whilst the DSPL competition has only existed since 2017\footnote{\url{https://athome.robocup.org/robocuphome-spl}}. All the software discussed in this paper is available open-source at GitHub\footnote{\url{https://github.com/tue-robotics}}, as well as various tutorials to assist with implementation. The main developments that resulted in the large lead at RoboCup 2022, and eventually the championship, are our central world model, discussed in Section \ref{sec:ed}, the generalized people recognition, discussed in Section \ref{sec:pers_recog}, the head display, discussed in Section \ref{ssec:display} and the new speech recognition system in Section \ref{sec:speech_rec}.


%Many parts of our software are interacting with the world-model. 
%Our world-model, \acrlong{ed}, is a database with 3D representations of objects. 
%This is described in section \ref{sec:ed}. 
%Other topics described in this paper are image recognition, pose detection, sound source localisation, Human-Robot interaction, Software sharing and community contributions.
%\\
%The previous years our focus has been on our own robots, AMIGO and SERGIO. 
%This year we are shifting our focus from our own robots, AMIGO and SERGIO to the Toyota HSR.
%This Team Description Paper is part of the qualification package for RoboCup 2019 in Sydney, Australia and describes the current status of the @Home activities of Tech United Eindhoven.
