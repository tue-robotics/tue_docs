Tech United Eindhoven\footnote{\url{http://www.techunited.nl}} (established 2005) is the RoboCup student team of Eindhoven University of Technology\footnote{\url{http://www.tue.nl}} (TU/e), which joined the ambitious @Home League in 2011. 
Multiple world vice-champion titles have been obtained in the OPL during previous years, and this year, whilst competing in the DSPL for the first time, the world championship title was finally claimed.
%The team has multiple World vice-champion titles to its name, two in the last three years, and is the current European vice-champion. 
%The robot soccer middle-size Tech United team has an even more impressive track record with four world championship titles.See the Tech United website for more results.
Tech United Eindhoven consists of (former) PhD and MSc. students and staff members from different departments within the TU/e. 
This year, these team members succesfully migrated the software from our TU/e built robots, AMIGO and SERGIO, to the Toyota HSR. 
 


%Many parts of our software are interacting with the world-model. 
%Our world-model, \acrlong{ed}, is a database with 3D representations of objects. 
%This is described in section \ref{sec:ed}. 
%Other topics described in this paper are image recognition, pose detection, sound source localisation, Human-Robot interaction, Software sharing and community contributions.
%\\
%The previous years our focus has been on our own robots, AMIGO and SERGIO. 
%This year we are shifting our focus from our own robots, AMIGO and SERGIO to the Toyota HSR.
%This Team Description Paper is part of the qualification package for RoboCup 2019 in Sydney, Australia and describes the current status of the @Home activities of Tech United Eindhoven.
