\label{ssec:conversation}
The \emph{conversation\_ engine}\footnote{\url{https://github.com/tue-robotics/conversation_engine}} bridges the gap between text input and an action planner (called action\_server). Text can be received from either Speech-to-Text or from a chat interface, like Telegram\texttrademark\hspace{0em}. The text is parsed according to a (Feature) Context Free Grammar, resulting in an action description in the form of a nested mapping. In the action description, (sub)actions and their parameters are filled in.
This may include references such as “it”.

Based on the action description, the action\_server tries to devise a sequence of actions and parameterize those with concrete object IDs.
%It may be that more information is needed (e.g. ``Place a coke on the dinner table'').
%The action then fails for this reason and then the conversation engine must interact with the user to obtain the missing information (e.g. ``Where should I pick that up?'').
To fill in missing information, the conversation\_engine engages with the user. When the user supplies more information, the additional input is parsed in the context of what info is missing. % and the current action description is extended and retried. To parse the additional information to fill in gaps of missing info, the conversation\_engine must know what field of missing information (e.g. `source-location' of a `grab' action) must be parsed according to what rules.
%The conversation\_engine is therefore parameterized with a mapping that links fields containing e.g. `location' to a rule in the grammar called `LOCATION'. 
Lastly, it keeps the user “informed” whilst actions are being performed by reporting on the current subtask.
