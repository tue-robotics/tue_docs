\label{ssec:keyboard}
\noindent The user interface modality as explained above has been extended to reduce the room for operator error by only presenting the user with a limited number of buttons in the Telegram app. This has been realized through Telegram's \emph{custom\_keyboards}\footnote{\url{https://github.com/tue-robotics/telegram_ros}} feature. This feature is especially useful when there are only a few options, like a predetermined selection of drinks, as shown in our RoboCup@Home 2019 Finals.

We have employed this custom keyboard to compose commands word-for-word (\emph{hmi\_telegram}\footnote{\url{https://github.com/tue-robotics/hmi_telegram}}). After a  user input has been received, either via text or previous buttons, for example “Bring me the ...”, the user is presented with only those words as options that might follow the input according to the grammar, eg. “apple”, “orange” etc. This process iterates until a full command has been composed.