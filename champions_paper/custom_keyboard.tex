\label{ssec:keyboard}
\noindent The user interface modality as explained above has been extended to reduce the room for operator error by only presenting the user with a limited number of buttons in the Telegram app. This has been realized through Telegrams \emph{custom\_ keyboards}\footnote{\url{https://github.com/tue-robotics/https://github.com/tue-robotics/telegram_ros/tree/rwc2019}} feature. This feature is especially useful if there are only a few options, such as when selecting from a predetermined selection of drinks, as has been shown in our finals during RoboCup 2019.

Since the competition, this feature has been employed to compose commands word-for-word. After the user has already entered, via text or previous buttons, for example “Bring me the ...” the user is presented with only those words that might follow that text according to the grammar, eg. “apple”, “orange” etc. This process iterates until a full command has been composed. This feature is called \emph{hmi\_ telegram}\footnote{\url{https://github.com/tue-robotics/hmi_telegram}}