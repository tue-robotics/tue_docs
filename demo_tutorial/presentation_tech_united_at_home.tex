\documentclass[a4paper,10pt]{article}
\usepackage{a4wide}
\usepackage[english]{babel}
\usepackage{amsmath}
\usepackage{graphicx}
\usepackage{subfigure}
\usepackage{hyperref}
\usepackage{float}
\usepackage{tabularx} 
\usepackage{hhline}
\usepackage{booktabs} 
\usepackage[table]{xcolor}
\usepackage{enumitem}
\usepackage{listings}
%\usepackage{array}
%\usepackage{subfig}
%\usepackage{siunitx}
%\usepackage{color}
%\usepackage{fancyvrb}
\usepackage{fancyhdr}
\pagestyle{fancy}
\usepackage{verbatim}
\fancyhf{}
\cfoot{Version 1.1}

\newcommand{\HRule}{\rule{\linewidth}{0.5mm}}
\numberwithin{equation}{section}
\numberwithin{figure}{section}
\numberwithin{table}{section}
\parindent=0pt

\begin{document}
\begin{titlepage}

\begin{center}
\vspace{15mm}
\end{center}

\begin{figure}[H]
\centering
\end{figure}

\begin{center}
\vspace{10mm}
{\Huge Presentatie @Home team Tech United}\\
\vspace{3mm}
{\Large Eindhoven University of Technology}\\
\vspace{10mm}
\vspace{3mm}
%\begin{figure}[H]
%\center
%\includegraphics[scale=0.6]{AMIGO.jpg}
%\end{figure}
\today
\end{center}

\vfill


\begin{tabular}{l l}
Version 1.1\\
\end{tabular}

\end{titlepage}

\section*{Hoe geef ik een demo met AMIGO?}
\textit{Met Tech United geven we regelmatig demonstraties met AMIGO en met onze voetbalrobots. Deze demo’s vinden plaats op veel verschillende locaties en evenementen. We hebben dus vaak een verschillend publiek dat kan vari\"eren van 5-jarige basisschoolleerlingen tot internationale hoogleraren. Het praatje dat je houdt tijdens de demo moet je dus ook op het publiek afstemmen. Bij basisschoolleerlingen treed je niet teveel in de technische details maar leg je in Jip \& Janneke taal uit hoe het werkt en bij techneuten kun je natuurlijk wel in detail treden. Ook moet je kijken of je publiek geboeid kunt houden. Kun je dat niet, ga dan geen langdradige verhalen vertellen maar sla wat over om het tempo erin te houden. Als je merkt dat het publiek erg geïnteresseerd en interactief is kun je de verhalen wat langer houden. Zorg er ook altijd voor dat je alles met overtuiging brengt. Weet je iets niet helemaal zeker, vertel het dan niet! Los van het publiek ziet de opbouw van een demo er eigenlijk altijd hetzelfde uit. Deze is als volgt:}

\section*{Introductie}
\textit{In de introductie verwelkom je het publiek en je introduceert jezelf en Tech United. Je vertelt wat Tech United is, wat we doen, wie we zijn en waarom we met robots bezig zijn:}\

Hallo, welkom allemaal. Welkom bij Tech United! Mijn naam is ..... en dit zijn mijn collega’s ..... en ...... \\

Als eerste zullen we even kort een samenvatting geven van wat TechUnited is en wat we doen. TechUnited is een team van (oud) studenten, PhD'ers en medewerkers van de Technische Universiteit Eindhoven. Kort gezegd houden wij ons bezig met ontwikkeling van robotica. Dit houd in dat we kennis van verschillende gebieden zoals werktuigbouwkunde, elektrotechniek en software science combineren om verschillende problemen aan te kunnen pakken.\\

Tech United doet wereldwijd mee aan verschillende toernooien, waaronder de Robocup. RoboCup is een jaarlijks terugkerend wereldkampioenschap voor robots die kunnen communiceren met en reageren op een constant veranderende omgeving. RoboCup is een opensource competitie wat betekent dat na elk toernooi alle gegevens van alle teams worden gedeeld. Ook worden de regels van RoboCup elk jaar uitgebreid. Dit zorgt ervoor dat de teams continu worden uitgedaagd en constant moeten innoveren om de concurrentie voor te blijven.\\

Tech United doet mee aan de Robocup met twee verschillende teams. Het ene team doet mee aan een competitie voor autonome voetbalrobots en het andere team doet mee aan de \@ Home League met een autonome zorg robot. Afgelopen jaar zijn beide teams tweede geworden op het wereldkampioenschap in Nagoya.\\

Vandaag zijn wij hier om jullie iets meer te vertellen over het @Home team van TechUnited. Het doel van dit team is om in de nabije toekomst zorgrobots beschikbaar te maken die de werkdruk op verplegend personeel kunnen verlichten. Zulke zorgrobots kunnen  ondersteuning bieden bij het schoonmaken, het tillen van pati\"enten en het toedienen van medicijnen. Hierdoor kunnen ouderen die dat graag willen zelf langer zelfstandig thuis wonen. 

\section*{AMIGO}
\textit{In het stukje over AMIGO wordt AMIGO voorgesteld en worden de belangrijkste en interessantste delen van AMIGO besproken. De inhoud van dit stukje moet naar eigen inzicht worden aangepast indien AMIGO niet mee is op een demo of indien de presentatie en de demo tegelijkertijd plaats vinden. }\\

Dan nu het moment waar jullie allemaal op hebben gewacht! Ik wil jullie graag voorstellen aan AMIGO! De naam AMIGO is een afkorting van Autonomous Mate for IntelliGent Operations. In het Nederlands betekent dit dat AMIGO een autonome robot vriend is die mensen kan helpen met het uitvoeren van abstracte taken. \\

Onze grote vriend hier is 1.5 meter groot en heeft zoals jullie kunnen zien net zoals ons 2 armen, wat zeker niet standaard is voor een robot. AMIGO heeft met zijn 80 kilo een vergelijkbaar gewicht met de gemiddelde Nederlandse man. Al moeten wij hem er toch af en toe aan herinneren dat hij gezond eet en genoeg sport!
Al is dat laatste af en toe moeilijk aangezien hij met zijn 4 batterijen maar 15 minuten actief kan zijn. Echter als hij een beetje lui is kan hij wel 30 minuten bezig blijven. Gelukkig heeft AMIGO wel 3 computers aan boord om al het moeilijke rekenwerk voor hem te doen.\\

\textit{Voor bepaald publiek is het leuk om de mensen naar de belangrijkste onderdelen van AMIGO te laten raden. Bij dit soort publiek is het van belang om de challenges te gebruiken als een opstapje.}\\

Nu jullie AMIGO hebben leren kennen is het tijd om er achter te komen wat AMIGO allemaal wel niet kan! Om punten te scoren op de wereldkampioenschappen moet AMIGO verschillende challenges succesvol kunnen uitvoeren. De challenges vinden plaats in een arena die lijkt op een gewone woonkamer. Het aantal challenges en de inhoud ervan veranderd elk jaar. We zullen hier een aantal voorbeelden van recente challenges geven.

\subsection*{Speech and person recognition}
De robot moet een groep personen vinden in de arena, vervolgens gaat het publiek rondom de robot staan en stelt vragen aan de robot. De vragen gaan over algemene kennis, maar er kunnen ook vragen over de objecten en de personen in de arena gesteld worden. Bijvoorbeeld: Hoeveel personen glimlachen er? Waar is de koelkast? Extra punten kunnen behaald worden door zich te richten tot de persoon die praat en bevestigen wie er praat. 

\subsection*{Help me carry}
Nadat de eigenaar van de robot terug komt van het winkelen, moeten de boodschappen in huis getild worden. De robot helpt hierbij, hij moet de operator volgen naar de auto, een tas pakken en naar binnen brengen. Binnen is er een tweede persoon aanwezig die komt helpen tillen, de robot moet de persoon begeleiden naar de auto.

\subsection*{Storing Groceries}
Nu de boodschappen binnen zijn moeten ze opgeruimd worden in huis. De boodschappen worden op tafel gelegd door de operator en de robot moet de boodschappen opruimen in de kast. Het in de kast leggen van de boodschappen moet wel logisch gebeuren, hier wordt onder verstaan dat het nieuwe pak koekjes naast het oude pak koekjes in de kast wordt gelegd.\

Voor de challenges zijn bijna alle producten bekend bij de robots, maar extra punten zijn te behalen met 'onbekende' objecten.  

\section*{Robot onderdelen}
\textit{Zoals gezegd is het voor bepaald publiek leuk om de mensen naar de belangrijkste onderdelen van AMIGO te laten raden. Je kunt het publiek eventueel ook wat helpen met het noemen van de onderdelen. Daarnaast is het van belang dat je de diepte van het technische aspect aanpast aan je publiek.}\\

\subsection*{Omniwielen}
De robot moeten zich natuurlijk door een ruimte kunnen verplaatsen. Wij mensen gebruiken daarvoor onze benen maar onze robot heeft daar speciale wielen voor aan boord. *Wijs naar de wielen* Dit zijn de zogenaamde omniwielen. Deze wielen worden in een richting aangedreven maar kunnen in alle andere richtingen wrijvingsloos bewegen. Dit wordt mogelijk gemaakt door de rubberen wieltjes op de omniwielen. AMIGO heeft 4 van deze omniwielen aan boord en door de aandrijving van de wielen slim te combineren kan hij elke kant op rijden. Bovendien kan hij om zijn as draaien. \\


Om te kunnen 'zien' zoals wij kunnen gebruikt AMIGO meerdere onderdelen.
\subsection*{Laser}
AMIGO is uitgerust met 2 lasers. De eerste laser zit op het torso *wijs het rode laser stripje aan op het torso* de andere laser zit onderaan de robot *wijs laser blokje aan*. Deze lasers maken, los van elkaar, een veld. Dus er is een laserveld op borst hoogte en een veld op scheen hoogte. Deze twee velden worden gebruikt om er voor te zorgen dat AMIGO nergens tegen aan botst. De lasers zijn op deze hoogtes geplaatst omdat de meeste objecten op een of beide hoogtes zichtbaar zijn. Echter kan AMIGO met deze laser nog niet echt goed zien. Hij kan objecten zien als ze aanwezig zijn, maar hij kan ze niet onderscheiden.

\subsection*{Kinect V2}
Om objecten te kunnen onderscheiden gebruikt AMIGO een Kinect *wijs kinect aan*. Jullie herkennen dit onderdeel van de robot misschien wel, dit onderdeel wordt namelijk ook gebruikt om spelletjes op de Xbox te kunnen spelen. AMIGO gebruikt de Kinect dus om objecten te kunnen onderscheiden. Een onderdeel hiervan is dat AMIGO mensen kan herkennen. Als iemand zich aan AMIGO heeft voorgesteld kan hij deze persoon vinden in een groep personen. Door gebruik te maken van deze Kinect kan AMIGO ook diepte zien en mensen volgen.

\subsection*{Wereldmodel}
Naast de lasers en de Kinect gebruikt AMIGO ook een wereldmodel om te kunnen zien. In tegenstelling tot de de lasers en de Kinect is het wereldmodel geen hardware, maar software. Dit betekent dat het wereldmodel geen fysieke vorm heeft maar dat het in zijn geheel in een computer zit. Zoals de naam al prijsgeeft is het wereldmodel een model van de wereld. Iets specifieker gezegd is het een model van de ruimte waardoor AMIGO zich moet bewegen. Dit is een van de weinige onderdelen van AMIGO die geheel niet autonoom is, wat dus betekent dat AMIGO grotendeels vertrouwt op zijn menselijke vrienden om dit model voor hem te maken. Echter is onze robotische vriend slim genoeg om kleine aanpassingen te maken aan het wereld model. Met behulp van zijn sensoren kan AMIGO 'zien' als bijvoorbeeld een tafel een beetje verschoven is. Voor iedere ruimte moet een nieuw model gemaakt worden, wat vrij veel werk oplevert. Mede met deze reden testen wij dan ook altijd in dezelfde ruimte.

\subsection*{Armen en Torso}
Alweer de laatste onderdelen van AMIGO waar we het over gaan hebben zijn de armen en het torso. AMIGO's torso kan omhoog en omlaag schuiven om bepaalde laag of juist hoog gelegen plekken te kunnen bereiken met zijn armen. En over zijn armen gesproken, AMIGO's armen lijken sprekend op menselijke armen. Waardoor hij bijna alles met zijn armen kan wat wij ook kunnen. Doordat AMIGO zoveel verschillende bewegingen kan maken met zijn armen kan hij een handpositie op verschillende manieren bereiken. *Doe even voor dat je een gelijke handpositie kan bereiken met verschillende armstanden* Doordat er verschillende mogelijkheden zijn om dezelfde positie te bereiken moet AMIGO van te voren goed nadenken en berekenen welke manier hij gaat gebruiken.
 

\section*{Afsluiting}
\textit{Na de demo vragen we of alles duidelijk was en of er nog vragen zijn. Na het beantwoorden van de eventuele vragen bedanken we het publiek voor de aandacht en wensen hen nog een fijne dag. Belangrijk is om aan te geven dat je ons kunt volgen op Facebook en Twitter. Zorg ook dat iedereen op YouTube gaat kijken naar hele wedstrijden, dit is namelijk veel interessanter dan wat je zojuist hebt laten zien!}

\end{document}