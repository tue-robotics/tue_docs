The \emph{ed\_localization}\footnote{\url{https://github.com/tue-robotics/ed_localization}} plugin implements \acrshort{amcl} based on a 2D render from the central world model. In order to navigate, a model of the environment is required. This model is stored in \acrshort{ed}. From this model, a planning representation is derived that enables using the model of the environment for navigation purposes.
\\
With use of the \emph{ed\_navigation} plugin\footnote{\url{https://github.com/tue-robotics/ed_navigation}}, an occupancy grid is derived from the world model and published. This grid can be used by a motion planner to perform searches in the configuration space of the robot.
\\
With the use of the \emph{cb\_base\_navigation} package\footnote{\url{https://github.com/tue-robotics/cb_base_navigation}} the robots are able to deal with end goal constraints. With use of a ROS service, provided by the \emph{ed\_navigation} plugin, an end goal constraint can be constructed w.r.t. a specific world model entity described by \acrshort{ed}. This enables the robot to not only navigate to poses but also to areas or entities in the scene. Navigation to an area is also shown in Figure \ref{fig:ed_segmentation}.
% as illustrated by Figure \ref{fig:ed_navigation_constraints}.
Somewhat modified versions of the local and global ROS planners available within \emph{move\_base} are used.
