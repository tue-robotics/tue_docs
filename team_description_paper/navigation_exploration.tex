The \emph{ed\_localization}\footnote{\url{https://github.com/tue-robotics/ed_localization}} plugin implements \acrshort{amcl} based on a 2D render of the central world model. 
\\
With use of the \emph{ed\_navigation} plugin\footnote{\url{https://github.com/tue-robotics/ed_navigation}}, an occupancy grid is derived from the world model and published. 
\\
With the use of the \emph{cb\_base\_navigation} package\footnote{\url{https://github.com/tue-robotics/cb_base_navigation}} the robots are able to deal with end goal constraints. 
The \emph{ed\_navigation} plugin allows to construct an end-gal constraint w.r.t. a world model entity in \acrshort{ed}. 
This enables the robot to also navigate to areas or entities in the scene, as well as waypoints. 
Navigation to an area is also shown in Figure \ref{fig:ed_segmentation}.
% as illustrated by Figure \ref{fig:ed_navigation_constraints}.
Somewhat modified versions of the local and global ROS planners available within \emph{move\_base} are used.
