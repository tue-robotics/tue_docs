In this paper, this year's main developments of Tech United Eindhoven have been discussed: 
\begin{itemize}
	\item Localization, navigation and object segmentation all benefit from the updating the pose of furniture objects by means of a novel fitting algorithm.
	\item Interaction with the robots is possible from a large variety of platforms due to the WebGUI.
	\item A Natural Language Interpreter allows an easier specification of commands to the robot and a robust speech recognition by excluding meaningless commands from the options that the robot is able to understand.
%	\item A new world model has been developed: Environment Descriptor. By combining semantic and volumetric information about the environment, this world model can be used not only for task planning and execution but also for motion planning.
%	\item Perception algorithms are implemented as `workers' on world model entities which allows for real-time classification of objects.
%	\item One of the benefits of using a volumetric world model is that predefined waypoints have become obsolete. Instead, a goal area can be defined depending on the task and the object at hand, which greatly robustifies navigation.
\end{itemize}
With these improvements, we hope to improve on last year's performance. We are looking forward to RoboCup~2016 in Leipzig!