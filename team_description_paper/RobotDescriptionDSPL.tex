\section{HSR's Software and External Devices}
% In this section briefly describe the software and hardware of the robot
We use a standard Toyota\texttrademark\hspace{0em} HSR robot. To differentiate our unit, we named it HERO. We wanted to link it's name to our AMIGO and SERGIO domestic service robots.

\noindent An overview of the software used by the Tech United Eindhoven @Home robots can be found in Table~\ref{tab:softwarespec}.
All our software is developed open-source at GitHub\footnote{\url{https://github.com/tue-robotics}}.
\begin{table}[H]
    \begin{center}
    \caption{Software overview}
    \label{tab:softwarespec}
    %\vspace{-0.1cm}
    \renewcommand{\arraystretch}{1.0}
    \setlength{\tabcolsep}{5pt}
        \begin{tabular}{p{0.3\textwidth} p{0.7\textwidth}}
            \toprule
            Operating system & Ubuntu 16.04 LTS Server\\

            Middleware & ROS Kinetic~\cite{Quigley2009}\\

            Simulation & Gazebo\\

            World model & \acrfull{ed}, custom \newline
            \url{https://github.com/tue-robotics/ed}\\

            Localization & Monte Carlo~\cite{Fox2003} using \gls{ed}, custom \newline \url{https://github.com/tue-robotics/ed_localization}\\

            SLAM & Gmapping\\

            Navigation & CB Base navigation
            \newline
            \url{https://github.com/tue-robotics/cb_base_navigation}
            \newline
            Global: custom A* planner\newline Local: modified ROS DWA~\cite{Fox1997}\\

            Arm navigation & MoveIt!\\

            Object recognition & image\_recognition\_tensorflow \newline
			\url{https://github.com/tue-robotics/image_recognition/tree/master/image_recognition_openface}\\

            People detection & Custom implementation using contour matching \newline
            \url{https://github.com/tue-robotics/ed_perception}
            \\
            Face detection \& recognition & image\_recognition\_openface \newline \url{https://github.com/tue-robotics/image_recognition/tree/master/image_recognition_openface} \\

            Speech recognition & Windows Speech Recognition\\
            
            Speech synthesis & Toyota\texttrademark \hspace{0em} Text-to-Speech\\
            Task executors & SMACH \newline
            \url{https://github.com/tue-robotics/tue_robocup}\\
            \bottomrule
        \end{tabular}
    \end{center}
\end{table}
\subsubsection{External Devices}
% Please describe in this section the external devices used by your robot. Consider the following example:
\textit{HERO relies on the following external hardware:}
\begin{itemize}
    \item Official Standard Laptop
    \item USB power speaker
    \item Gigabit Ethernet Switch
    \item Wi-Fi adapter
\end{itemize}

\subsubsection{Cloud Services}
% Please describe in this section the Cloud Services and online software used by your robot. Consider the following example:

\textit{HERO connects the following cloud services:} None
%\begin{itemize}
%	\item Localization and mapping: Geolocalization system.
%	\item Navigation: Navigator
%	\item Speech recognition: All-purpose recognizer.
%\end{itemize}

