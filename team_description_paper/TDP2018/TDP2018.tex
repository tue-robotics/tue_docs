
%%%%%%%%%%%%%%%%%%%%%%% file typeinst.tex %%%%%%%%%%%%%%%%%%%%%%%%%
%
% Template author: Mauricio Matamoros
% Updated: July 3, 2017
% Contact: mauricio@robocupathome.org
%
% This is the LaTeX source for the TDPTemplate using
% the LaTeX document class 'llncs.cls' Springer LNAI format
% used in the RoboCup Symposium submissions.
% http://www.springer.com/computer/lncs?SGWID=0-164-6-793341-0
%
% It may be used as a template for your own TDP - copy it
% to a new file with a new name and use it as the basis
% for your Team Description Paper
%
% NB: the document class 'llncs' has its own and detailed documentation, see
% ftp://ftp.springer.de/data/pubftp/pub/tex/latex/llncs/latex2e/llncsdoc.pdf
%
% Remark: Last page with specs won't be included in Camera ready TDP's.
%
%%%%%%%%%%%%%%%%%%%%%%%%%%%%%%%%%%%%%%%%%%%%%%%%%%%%%%%%%%%%%%%%%%%
%
% TU/e update: 27 oct 2017 - HvR

\documentclass[runningheads,a4paper]{llncs}
%\usepackage[left=48mm,right=46mm]{geometry}
\usepackage[left=32mm,right=31mm]{geometry}

\usepackage[utf8]{inputenc}
\usepackage{amssymb}
\setcounter{tocdepth}{3}
\usepackage{url}
\usepackage{float}
\usepackage{amsmath}
\usepackage{graphicx}
\usepackage{wrapfig}
\usepackage{fancyhdr}
%\usepackage{titling}
%\usepackage{lipsum}  		% a garbage package you don't need except to create examples.

\newcommand{\robospecs}{%
	\newpage%
	\pagenumbering{gobble}%
	\pagestyle{fancy}%
	\fancyhf{}%
	\chead{$|$}
	\rhead{\footnotesize Robot Description}%
	\lhead{\footnotesize Tech United Eindhoven}%
	\rfoot{Robot software and hardware specification sheet}%
}

\newcommand{\BnL}[1][1em]{ \includegraphics[width=#1]{images/bnl.jpg} } 

% added HvR - 27oct2017 - from last year's (2017) template
\usepackage[english]{babel}		% input language for hyphens
%\usepackage{caption}
%\usepackage{subcaption}
%\captionsetup{compatibility=false}
\usepackage{listings}
\usepackage{enumitem}
\usepackage{textcomp}
%\usepackage{hyperref}

% fonts
\usepackage[T1]{fontenc}		% more glyphs and a
\usepackage{lmodern}			% better looking font

% *** MORE GRAHPICS ***
\usepackage[usenames,dvipsnames]{color}


%%%%%%%%%%%%%%%%%%%%%%%%%%%%%%%%%%%%%%%%%%%%%%%%%%%%%%%%%%%%%%%%%%%
\usepackage{booktabs}   			% For tables (toprule, midrule, bottomrule)
\usepackage{todonotes}			% should be defined after the color package
%%%%%%%%%%%%%%%%%%%%%%%%%%%%%%%%%%%%%%%%%%%%%%%%%%%%%%%%%%%%%%%%%%%

%%%%%%%%%%%%%%%%%%%%%%%%%%%%%%%%%%%%%%%%%%%%%%%%%%%%%%%%%%%%%%%%%%%
% *** PATHS ***
\makeatletter
\def\input@path{{Figures/}		
				}
\makeatother
\graphicspath{{Figures/}
				}
%%%%%%%%%%%%%%%%%%%%%%%%%%%%%%%%%%%%%%%%%%%%%%%%%%%%%%%%%%%%%%%%%%%

%%%%%%%%%%%%%%%%%%%%%%%%%%%%%%%%%%%%%%%%%%%%%%%%%%%%%%%%%%%%%%%%%%%

% Acronym definitions
\usepackage[acronym]{glossaries}
\newacronym{ed}{ED}{Environment Descriptor}
\newacronym{amcl}{AMCL}{Adaptive Monte Carlo Localization}
\newacronym{gui}{GUI}{Graphical User Interface}
%\newacronym{spl}{SPL}{Standard Platform League}
\newacronym{fcfg}{FCFG}{feature context free grammar}
\newacronym{ros}{ROS}{Robot Operating System}
\newacronym{wire}{WIRE}{World Information for Robotic Environments}
\newacronym{cnn}{CNN}{Convolution Neural Networks}

% shorthand definitions
\newcommand{\eg}{\emph{e.g.}}						% Exemplum gratia
\newcommand{\goal}{\mathcal{G}}						% Goal area
\newcommand{\goallc}{\mathcal{G}_{\mathrm{lc}}}			% Subset of goal area with costs below threshold cmin
\newcommand{\goalhc}{\mathcal{G}_{\mathrm{hc}}}			% Subset of goal area with costs above threshold cmin
\newcommand{\ie}{\emph{i.e.}}							% Id est
%%%%%%%%%%%%%%%%%%%%%%%%%%%%%%%%%%%%%%%%%%%%%%%%%%%%%%%%%%%%%%%%%%%

%\usepackage{hyperref}
%%%%%%%%%%%%%%%%%%%%%%%%%%%%%%%%%%%%%%%%%%%%%%%%%%%%%%%%%%%%%%%%%%%%%%%%%%%%%%%%%%%%
%
% Title
%
%%%%%%%%%%%%%%%%%%%%%%%%%%%%%%%%%%%%%%%%%%%%%%%%%%%%%%%%%%%%%%%%%%%%%%%%%%%%%%%%%%%%
\begin{document}

\setlength{\headheight}{22pt}

\title{Tech United Eindhoven @Home \\2018 Team Description Paper}

\author{M.F.B.~van~der~Burgh , J.J.M.~Lunenburg, R.P.W.~Appeldoorn, R.W.J.~Wijnands, 
T.T.G.~Clephas, M.J.J.~Baeten, L.L.A.M.~van~Beek, R.A.~Ottervanger, 
S.~Aleksandrov, T.~Assman, K.~Dang, J.~Geijsberts, L.G.L.~Janssen, 
H.W.A.M.~van~Rooy, A.T. Hofkamp and M.J.G.~van~de~Molengraft}

\institute{Eindhoven University of Technology,\\
Den Dolech 2, P.O. Box 513, 5600 MB Eindhoven, The Netherlands\\
\texttt{http://www.techunited.nl, techunited@tue.nl, https://github.com/tue-robotics}}

\authorrunning{Tech United Eindhoven}

%\author{Team Leader \and Team Members }
%\institute{[Institute name and direction here], \\
%\texttt{http://devoted-web-site.url}}

\maketitle
%%%%%%%%%%%%%%%%%%%%%%%%%%%%%%%%%%%%%%%%%%%%%%%%%%%%%%%%%%%%%%%%%%%%%%%%%%%%%%%%%%%%
%
% Abstract
%
%%%%%%%%%%%%%%%%%%%%%%%%%%%%%%%%%%%%%%%%%%%%%%%%%%%%%%%%%%%%%%%%%%%%%%%%%%%%%%%%%%%%
%
\begin{abstract}

%TODO list:\\
%Sound source localisation\\
%OpenPose\\
%Plans for this year?\\
%Clean-up of old stuff, reduce size. Increase size new parts.\\
%SERGIO in/out?\\
%New template: \url{https://github.com/RoboCupAtHome/TDPTemplate/tree/template2018}\\
%8 pages\\

%Video:\\
%Sound source localisation\\
%OpenPose\\
%Remove old stuff.\\
%Remove ugly video stuff\\



TODO: UPDATE \\
This paper provides an overview of the main developments of the Tech United Eindhoven RoboCup@Home team. Tech United uses an advanced world modeling representation system called the Environment Descriptor that allows straight forward implementation of localization, navigation, exploration, object detection \& recognition, object manipulation and robot-robot cooperation skills. Recent developments are improved object detection via deep learning methods, a generic GUI for different user levels, improved speech recognition, improved natural language interpretation and sound source localization.

\end{abstract}
%
%
%%%%%%%%%%%%%%%%%%%%%%%%%%%%%%%%%%%%%%%%%%%%%%%%%%%%%%%%%%%%%%%%%%%%%%%%%%%%%%%%%%%%%
%
\section{Introduction}
Tech United Eindhoven\footnote{\url{http://www.techunited.nl}} is the RoboCup student team of Eindhoven University of Technology\footnote{\url{http://www.tue.nl}} that (since 2005) successfully competes in the robot soccer Middle Size League (MSL) and later (2011) joined the ambitious @Home League. The Tech United @Home team is the vice champion of RoboCup 2017 in Nagoya, Japan and the vice European Champion of the 2017 RoboCup German Open. The robot soccer middle-size Tech United team has an even greater track record with 3 world championship titles. See the Tech United website for more results. Tech United Eindhoven consists of (former) PhD and MSc. students and staff members from different departments within the Eindhoven University of Technology.
\\\\
This Team Description Paper is part of the qualification package for RoboCup 2018 in Montreal, Canada and describes the current status of the @Home activities of Tech United Eindhoven. TODO: UPDATE. The main achievement of our long-term development is our generic world model \acrshort{ed}. Recent developments are improved object detection via deep learning methods, a generic GUI for different user levels, improved speech recognition and improved natural language interpretation. 

\section{\acrfull{ed}}
\label{sec:ed}
The TU/e \acrfull{ed} is a \acrfull{ros} based 3D geometric, object-based world representation system for robots. ED is a database system that structures multi-modal sensor information and represents this such that it can be utilized for robot localisation, navigation, manipulation and interaction. Figure \ref{fig:ed} shows a schematic overview of ED.

ED has been used on our robots in the OPL since 2012 and was also used this year in the DSPL. Previous developments have focused on making ED platform independent, as a result ED has been used on the PR2, Turtlebot, Dr. Robot systems (X80), as well as on multiple other @Home robots.
\begin{figure}[h]
    %\vspace{-0.3cm}
	\includegraphics[width = 0.9\linewidth]{Figures/ed_overview}
    %\vspace{-1em}
	\caption{Schematic overview of TU/e Environment Descriptor. Double sided arrows indicate that the information is shared both ways, one sided arrows indicate that the information is only shared in one direction.}
	\label{fig:ed}
    %\vspace{-0.5cm}
\end{figure}
ED is a single re-usable environment description that can be used for a multitude of desired functionalities such as object detection, navigation and human machine interaction. Improvements in ED reflect in the performances of the separate robot skills, as these skills are closely integrated in ED. It omits updating and synchronization of multiple world models. Currently, different \acrshort{ed} plug-ins exist that enable robots to localize themselves, update positions of known objects based on recent sensor data, segment and store newly encountered objects and visualize all this in RViz and through a web-based \acrshort{gui}, as illustrated in Figure \ref{fig:gui_actions}.
\begin{figure}[h]
\centering
    %\vspace{-0.3cm}
	\includegraphics[width = 0.8\linewidth]{Figures/ed_segmentation_hsr}
    %\vspace{-0.5em}
	\caption{A view of the world model created with \acrshort{ed}. The figure shows the occupancy grid as well as classified objects recognized on top of the cabinet.}
	\label{fig:ed_segmentation}
    %\vspace{-0.5cm}
\end{figure}


\subsection{Localization, Navigation and Exploration}
The \emph{ed\_localization}\footnote{\url{https://github.com/tue-robotics/ed_localization}} plugin implements \acrshort{amcl} based on a 2D render of the central world model. 
\\
With use of the \emph{ed\_navigation} plugin\footnote{\url{https://github.com/tue-robotics/ed_navigation}}, an occupancy grid is derived from the world model and published. 
\\
With the use of the \emph{cb\_base\_navigation} package\footnote{\url{https://github.com/tue-robotics/cb_base_navigation}} the robots are able to deal with end goal constraints. The \emph{ed\_navigation} plugin allows to construct an end-gal constraint w.r.t. a world model entity in \acrshort{ed}. This enables the robot to also navigate to areas or entities in the scene, as well as waypoints. Navigation to an area is also shown in Figure \ref{fig:ed_segmentation}.
% as illustrated by Figure \ref{fig:ed_navigation_constraints}.
Somewhat modified versions of the local and global ROS planners available within \emph{move\_base} are used.


\subsection{Object detection}
ED enables integrating sensors through the use of the plugins present in the \textit{ed\_sensor\_integration} package.
Two different plugins exist:
\begin{enumerate}
\item \emph{laser\_plugin}: Enables tracking of 2D laser clusters. This plugin can be used to track dynamic obstacles such as humans.
\item \emph{kinect\_plugin}: Enables world model updates with use of data from a RGBD camera. This plugin exposes several ROS services that realize different functionalities:
\begin{enumerate}[label=(\alph*)]
\item \emph{Segment}: A service that segments sensor data that is not associated with other world model entities. Segmentation areas can be specified per entity in the scene. This allows to segment object `on-top-of’ or ‘in’ a cabinet. All points outside the segmented area are ignore for segmentation.
\item \emph{FitModel}: A service that fits the specified model in the sensor data of a RGBD camera. This allows updating semi-static obstacles such as tables and chairs.
\end{enumerate}
\end{enumerate}

The \emph{ed\_sensor\_integration} plugins enable updating and creating entities. However, new entities are classified as unknown entities. Classification is done in \emph{ed\_perception} plugin\footnote{\url{https://github.com/tue-robotics/ed_perception}} package. 

\subsection{Object grasping, moving and placing}
The system architecture developed for object manipulation is focused on grasping. In the implementation, its input is a specific target entity in \acrshort{ed}, selected by a Python executive and the output is the grasp motion joint trajectory.
Figure \ref{fig:grasping_pipeline} shows the grasping pipeline.
\begin{figure}[H]
    \centering
    %\vspace{-0.3cm}
	\includegraphics[width = 1\linewidth]{Figures/grasping_pipeline}
    %\vspace{-1em}
	\caption{Custom grasping pipeline base on \acrshort{ed}, MoveIt and a separate grasp point determination and approach vector node.}
	\label{fig:grasping_pipeline}
    %\vspace{-0.5cm}
\end{figure}
MoveIt! is used to produce joint trajectories over time, given the current configuration, robot model, \acrshort{ed} world model (for collision avoidance) and the final configuration.
%\acrshort{ed} provides collision meshes to MoveIt! so it can plan paths that avoid obstacles.
%Finally, the trajectories are sent to the reference interpolator which sends the trajectories either to the controllers or the simulated robot.
\\
%The grasping pipeline is extended with an empty spot designator and grasping pose determination. The empty spot designator searches in an area, like `on-top-of'of the dinner table, for an empty spot to place an object by using the occupied area by other objects in this area.
\indent The grasp pose determination uses the information about the position and shape of the object in \acrshort{ed} to determine the best grasping pose.
The grasping pose is a vector relative to the robot.
An example of the determined grasping pose is shown in Figure \ref{fig:grasping_pose_determination}.
Placing an object is approached in a similar manner to grasping, except for that when placing an object, \acrshort{ed} is queried to find an empty placement pose.
\begin{figure}[H]
   \centering
   %\vspace{-0.3cm}
   \includegraphics[width = 0.8\linewidth]{Figures/grasp_point_determination}
    %\vspace{-1em}
	\caption{Grasping pose determination result for a cylindric object with TU/e built robot AMIGO. It is unpreferred to grasp the object from behind.}
	\label{fig:grasping_pose_determination}
    %\vspace{-0.5cm}
\end{figure}


\section{Image Recognition}
\input{object_recognition.tex}

\section{Sound source localization}
%To perform proper speech recognition, knowing the direction of the sound is important to capture the sound source properly.
We localize the sound source by determining the \acrfull{doa} using a Matrix Creator\footnote{\url{https://creator.matrix.one}} board.
%depicted in Figure \ref{fig:matrix_one}.
%\begin{figure}[h]
%    \centering
%    %\vspace{-0.3cm}
%	\includegraphics[width = 0.5\linewidth]{Figures/ssl_Matrix_creator.png}
%    %\vspace{-1em}
%	\caption{Matrix One Creator board.}
%	\label{fig:matrix_one}
%    %\vspace{-0.5cm}
%\end{figure}
%The detection is done by first calculating the time cross correlation between four pairs of opposing microphones.
%Second, the microphone pair with the lowest phase shift w.r.t. the opposing microphone is selected as being perpendicular to the source.
%Finally, the direction of the source can be determined by combining this information with the energy level of the microphones.
The detection is done by cross-correlating between pairs of opposing microphone, combined with finding the two microphones with the lowest mutual phase shift and using the energy level of the microphones.
%This upgrade compared to the stock DOA code of the Matrix Creator has been pushed back to their repositories.
This improvement is contributed back into the upstream repository.
%Our software for the DOA detection is available on GitHub\footnote{\url{https://github.com/tue-robotics/matrix-creator-hal}}, as well as a ROS package\footnote{\url{https://github.com/tue-robotics/matrix_creator_ros}} that exposes the DOA detections via a pose topic.
The \acrshort{doa} is published as a ROS Pose. Because the DSPL doesn't allow hardware changes to the robot, this software needs to be re-implemented for the Toyota HSR.



%\subsection{Reasoning}
%\input{reasoning.tex}

%\newpage
\section{Human-Robot Interface}
\label{ssec:webgui}
In order to interact with the robot, apart from speech, we have designed a web-based \gls{gui}. This interface uses HTML5\footnote{\url{https://github.com/tue-robotics/tue_mobile_ui}} with the Robot API written in Javascript and we host it on the robot itself.
%This allows multiple users on different platforms (\eg\ Android, iOS) to access functionalities of the robot. The interface is implemented in JavaScript with AngularJS and it offers a graphical interface to the Robot API\footnote{\url{https://github.com/tue-robotics/robot-api}} which exposes all the functionality of the robot.
%\begin{figure}[h]
%    \centering
%	\includegraphics[width=0.9\linewidth]{Figures/webgui_architecture}
%    %\vspace{-0.5em}
%	\caption{
%		Overview of the WebGUI architecture.
%		A webserver that is hosting the \protect\gls{gui} connects this %Robot API to a graphical interface that is offered to multiple clients on %different platforms.}
%	\label{fig:webgui_architecture}
%\end{figure}
%Figure~\ref{fig:gui_actions} gives an example of various user interactions that are possible with the \gls{gui} and the different commands that can be given to the robot while interacting with the virtual scene.

\begin{figure}[H]
	\includegraphics[width=\linewidth]{Figures/gui_actions}
	\caption{
		Illustration of the 3D scene of the WebGUI with AMIGO.
		User can long-press objects to open a menu from which actions on the object can be triggered
%		Users can interact with use of the menu that appears when long pressing an object in the scene.
%		On the left figure, the user commands the robot to inspect the selected object, which is the `cabinet'.
%		When the robot has inspected the `cabinet', it has found entities on top of it.
%		In the middle figure a grasp command is given to the robot to pick up an object from the cabinet.
%		The last figure show the robot executing that action.
		}
	\label{fig:gui_actions}

\end{figure}
%Figure \ref{fig:webgui_architecture} gives an overview of the connections between these components and 
\noindent Figure \ref{fig:gui_actions} represents an instance of the various interactions that are possible with the Robot API.


%\section{Robot Descriptions}
%\subsection{Robot Hardware Descriptions}
%\input{hardware.tex}
%
%\subsection{Robot Software Description}
%\input{software.tex}

\subsection{Re-usability of the system for other research groups}
Tech United takes great pride in creating and maintaining open-source software and hardware to accelerate innovation. Tech United initiated the Robotic Open Platform website\footnote{\url{http://www.roboticopenplatform.org}}, to share hardware designs. All our software is available on GitHub\footnote{\url{https://github.com/tue-robotics}}. All packages include documentation and tutorials.
The finals of Robocup@Home 2022\footnote{\url{https://tinyurl.com/TechUnited2022AtHomeFinals}} demonstrates all the capabilities of HERO, as described in the previous sections.
Tech United and its scientific staff have the capacity to co-develop (15+ people), maintain and assist in resolving questions.


\subsection{Community Outreach and Media}
The Tech United team carries out many promotional activities for children to promote technology and innovation. These activities are carried out by separate teams of student assistants. Tech United often visits primary and secondary schools, public events, trade fairs and has regular TV performances. In 2015 and 2016 combined, 100+ demos were given and an estimated 50k people were reached through live interaction.
Tech United also has a very active (\href{www.techunited.nl}{website}, and interacts on many social media like: \href{https://www.facebook.com/techunited}{Facebook}, \href{https://www.youtube.com/user/TechUnited}{YouTube}, \href{https://twitter.com/TechUnited}{Twitter} and \href{https://www.flickr.com/photos/techunited/}{Flickr}. Our robotics videos are often shared on the IEEE video Friday website.


%%%%%%%%%%%%%%%%%%%%%%%%%%%%%%%%%%%%%%%%%%%%%%%%%%%%%%%%%%%%%%%%%%%%%%%%%%%%%%%%%%%%%
%%
%% Bibliography\usepackage{graphicx}
%%
%%%%%%%%%%%%%%%%%%%%%%%%%%%%%%%%%%%%%%%%%%%%%%%%%%%%%%%%%%%%%%%%%%%%%%%%%%%%%%%%%%%%%

\section*{Bibliography}
\bibliographystyle{unsrt}
\bibliography{refs}


%%%%%%%%%%%%%%%%%%%%%%%%%%%%%%%%%%%%%%%%%%%%%%%%%%%%%%%%%%%%%%%%%%%%%%%%%%%%%%%%%%%%%
%%
%% Robot Specifications
%%
%%%%%%%%%%%%%%%%%%%%%%%%%%%%%%%%%%%%%%%%%%%%%%%%%%%%%%%%%%%%%%%%%%%%%%%%%%%%%%%%%%%%%
%
%\robospecs
%%\section{HSR's Software and External Devices}
% In this section briefly describe the software and hardware of the robot
We use a standard Toyota\texttrademark\hspace{0em} HSR robot. To differentiate our unit, we named it HERO. We wanted to link it's name to our AMIGO and SERGIO domestic service robots.

\noindent An overview of the software used by the Tech United Eindhoven @Home robots can be found in Table~\ref{tab:softwarespec}.
All our software is developed open-source at GitHub\footnote{\url{https://github.com/tue-robotics}}.
\begin{table}[H]
    \begin{center}
    \caption{Software overview}
    \label{tab:softwarespec}
    %\vspace{-0.1cm}
    \renewcommand{\arraystretch}{1.0}
    \setlength{\tabcolsep}{5pt}
        \begin{tabular}{p{0.3\textwidth} p{0.7\textwidth}}
            \toprule
            Operating system & Ubuntu 20.04 LTS Server\\

            Middleware & ROS Noetic~\cite{Quigley2009}\\

            Simulation & Gazebo\\

            World model & \acrfull{ed}, custom \newline
            \url{https://github.com/tue-robotics/ed}\\

            Localization & Monte Carlo~\cite{Fox2003} using \gls{ed}, custom \newline \url{https://github.com/tue-robotics/ed_localization}\\

            SLAM & Gmapping\\

            Navigation & CB Base navigation
            \newline
            \url{https://github.com/tue-robotics/cb_base_navigation}
            \newline
            Global: custom A* planner\newline Local: modified ROS DWA~\cite{Fox1997}\\

            Arm navigation & MoveIt!\\

            Object recognition & Inception based custom DNN~\cite{GoogleNet2015} \newline
			\url{https://github.com/tue-robotics/image_recognition}\\

            People detection & Custom implementation using contour matching \newline
            \url{https://github.com/tue-robotics/people_recognition}
            \\
            Face detection \& recognition & OpenFace~\cite{amos2016openface} \newline \url{https://github.com/tue-robotics/image_recognition} \\

            Speech recognition & Windows Speech Recognition, PicoVoice \newline
            \url{https://github.com/tue-robotics/picovoice_ros.git} \\
            
            Speech synthesis & Toyota\texttrademark \hspace{0em} Text-to-Speech\\
            Task executors & SMACH \newline
            \url{https://github.com/tue-robotics/tue_robocup}\\
            \bottomrule
        \end{tabular}
    \end{center}
\end{table}
\subsubsection{External Devices}
% Please describe in this section the external devices used by your robot. Consider the following example:
\textit{HERO relies on the following external hardware:}
\begin{itemize}
    \item Official Standard Laptop
    \item USB power speaker
    \item Gigabit Ethernet Switch
    \item Wi-Fi adapter
\end{itemize}

\subsubsection{Cloud Services}
% Please describe in this section the Cloud Services and online software used by your robot. Consider the following example:

\textit{HERO does not use any cloud services}
%\begin{itemize}
%	\item Localization and mapping: Geolocalization system.
%	\item Navigation: Navigator
%	\item Speech recognition: All-purpose recognizer.
%\end{itemize}


\newpage
%%\section*{Amigo's Hardware Description}
% In this section briefly describe the software and hardware of the robot

\setlength\intextsep{0pt}
\begin{wrapfigure}[15]{r}{0.3\textwidth}
	\centering
	\includegraphics[width=0.3\textwidth]{amigo}
	\caption{The Amigo Robot}
	\label{fig:amigo}
\end{wrapfigure}

AMIGO (Autonomous Mate for Intelligent Operations, see Fig.~\ref{fig:amigo}) has competed in RoboCup@Home since 2011. Its design is based on a Middle Size League soccer robot, equipped with two Philips\texttrademark \hspace{0em} Experimental Robotic Arms mounted on an extensible upper body. Based on our experiences with AMIGO, SERGIO (Second Edition Robot for Generic Indoor Operations, has been developed. The main differences with AMIGO are the use of Mecanum wheels which are compliantly suspended, the torso with two degrees of freedom and the modular setup. The core specifications of AMIGO are shown in Table~\ref{tab:hardwarespec}. More details about the robots are on the Robotic Open Platform\footnote{\texttt{http://www.roboticopenplatform.org/}}, where all CAD drawings, electrical schematics and CAD files are published. SERGIO will not enter the competition this year. \\

\begin{table}[H]
    \begin{center}
    \caption{Core specifications of AMIGO}
    \label{tab:hardwarespec}
    \renewcommand{\arraystretch}{1.0}
    \setlength{\tabcolsep}{5pt}
        \begin{tabular}{p{0.2\textwidth} p{0.4\textwidth}}
            \toprule
            & AMIGO \\
            \midrule
            Name & Autonomous Mate for IntelliGent Operations \\
            Base & Fully holonomic omni-wheel platform  \\
            Torso & 1 vertical DoF using a ball screw \\
            Manipulators & 2 7-DoF Philips\texttrademark \hspace{0em} Experimental Robotic Arms \\
            Neck & Pan-tilt unit using two Dynamixel RX-64 servo actuators \\
            Head & Microsoft Kinect\texttrademark \hspace{0em} for XBox 360\texttrademark \\
            External devices & Wireless emergency button \\
            Dimensions & Diameter: $0.75\ \mathrm{m}$, height: $\pm1.5\ \mathrm{m}$ \\
            Weight & $\pm84\ \mathrm{kg}$ \\
            Additional sensors & Hokuyo UTM-30LX laser range finder on base and torso\\
            Microphone & R{\O}DE Videomic and Matrix Creator\\
            Batteries & $4\times$ Makita $24\ \mathrm{V},\ 3.3\ \mathrm{Ah}$ \\
            Computers & $3\times$ AOpen Mini PC with Core-i7 processor and $8\ \mathrm{GB}$ RAM and NVidia Jetson TX2 \\
            \bottomrule
        \end{tabular}
    \end{center}
\end{table}

\newpage
\section*{AMIGO's Software Description}
% Please describe in this section the software you are using to control your robot. Consider the following example:

An overview of the software used by the Tech United Eindhoven @Home robots is shown in Table~\ref{tab:softwarespec}.
All our software is developed open-source on GitHub\footnote{\url{https://github.com/tue-robotics}}.
\\\newline
Some \textit{image\_recognition} packages are released into the ROS Kinetic distribution and can be installed with use of \textit{apt}.\\


\begin{table}[H]
    \begin{center}
    \caption{Software overview of Amigo.}
    \label{tab:softwarespec}
    %\vspace{-0.1cm}
    \renewcommand{\arraystretch}{1.0}
    \setlength{\tabcolsep}{5pt}
        \begin{tabular}{p{0.3\textwidth} p{0.7\textwidth}}
            \toprule
            Operating system & Ubuntu 16.04 LTS Server\\

            Middleware & ROS Kinetic~\cite{Quigley2009}\\

            Low-level control software & Orocos Real-Time Toolkit~\cite{Bruyninckx2001}\newline
            \url{https://github.com/tue-robotics/rtt_control_components}
            \\

            Simulation & Custom kinematics + sensor simulator \newline
            \url{https://github.com/tue-robotics/fast_simulator}
            \\

            World model & \acrfull{ed}, custom \newline
            \url{https://github.com/tue-robotics/ed}\\

            Localization & Monte Carlo~\cite{Fox2003} using \gls{ed}, custom \newline \url{https://github.com/tue-robotics/ed_localization}\\

            SLAM & Gmapping package \newline \url{http://wiki.ros.org/gmapping}\\

            Navigation & CB Base navigation
            \newline
            \url{https://github.com/tue-robotics/cb_base_navigation}
            \newline
            Global: custom A* planner\newline Local: modified ROS DWA~\cite{Fox1997}\\

            Arm navigation & Custom implementation using MoveIt and Orocos KDL\newline
            \url{https://github.com/tue-robotics/tue_manipulation}
            \\

            Object recognition & Tensorflow ROS \newline
			\url{https://github.com/tue-robotics/image_recognition/tree/master/tensorflow_ros}\\

            People detection & Custom implementation using contour matching \newline
            \url{https://github.com/tue-robotics/ed_perception}
            \\
            Face detection \& recognition & Openface ROS \newline \url{https://github.com/tue-robotics/image_recognition/tree/master/openface_ros} \\

            Speech recognition & Dragonfly + Windows\texttrademark \hspace{0em} Speech Recognition \newline
            \url{https://github.com/tue-robotics/dragonfly_speech_recognition}\\
            Speech synthesis & Philips\texttrademark \hspace{0em} Text-to-Speech\\
            Task executors & SMACH \newline
            \url{https://github.com/tue-robotics/tue_robocup}\\
            \bottomrule
        \end{tabular}
    \end{center}
\end{table}

\section*{External Devices}
% Please describe in this section the external devices used by your robot. Consider the following example:

\textit{AMIGO relies on the following external hardware:}

\begin{itemize}
	\item Tyro 2-channel wireless emergency stop (\url{https://www.tyroremotes.nl})
	\item Apple iPad for the web GUI (\url{https://www.apple.com})
\end{itemize}

\section*{Cloud Services}
% Please describe in this section the Cloud Services and online software used by your robot. Consider the following example:

\textit{AMIGO connects the following cloud services:}
\begin{itemize}
   \item Skybiometry face detection (\url{https://skybiometry.com/})
\end{itemize} 
\subsection{Robot Hardware Descriptions}
\input{hardware.tex}
\newpage
\subsection{Robot Software Description}
\input{software.tex}
%%\input{RobotDescriptionSSPL}
%
%\nocite{*}
\end{document} 