In the previous section, our approach to people recognition is explained. This recognition includes information about the posture of each 3D skeleton. Once the people information is inserted into the world model, additional properties can be added to the persons that take also other entities in the world model into account, e.g. `is\_pointing\_at\_entity'. This information is used by the toplevel state machines to implement challenges.
However an additional check based on spatial queries is inserted to ensure that the correct operator is found. By using such a query it is possible to filter out people based on their location. Finally, to determine at which entity the operator is pointing, ray-tracing is implemented. Figure \ref{fig:ray_trace} shows an example of the ray-tracing.

\begin{figure}[H]
	\centering
    \includegraphics[width=0.8\linewidth]{ed_ray_trace2}
	\caption{Ray-tracing based on pose detection with AMIGO.}
	\label{fig:ray_trace}
\end{figure}
